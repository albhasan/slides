% NOTE:
%" latexmk -shell-escape -pvc slides.tex # Watches and compiles on each change.
% latexmk -c slides.tex   # Clean the temporal files.

% NOTE:
% the minted package doesn't play well with the bibliography!

\documentclass[aspectratio=169]{beamer}

\setbeamertemplate{footline}[frame number]

\usepackage{caption}
\usepackage{graphicx}
\usepackage{hyperref}

\hypersetup{
    colorlinks=true,
    linkcolor=blue,
    filecolor=magenta,
    urlcolor=cyan,
    pdftitle={Overleaf Example},
    pdfpagemode=FullScreen,
}


\captionsetup[figure]{labelformat=empty}

\title{Methods for estimating the peak season in time series data}


\author{Alber S\'{a}nchez \href{mailto:alber.ipia@inpe.br}
{alber.ipia@inpe.br}\newline
Guilherme Mataveli}
\institute{
  \includegraphics[width=4cm,keepaspectratio]{logos/trees-color-h_2.png}
  \includegraphics[width=1.8cm,keepaspectratio]
  {logos/logoinpe-azul-menor.png} \\
  Research assistant - TreesLab\\National Institute for Space Research - INPE\\
  Brazil
}
\date{\today}

\begin{document}

\frame{\titlepage}

\begin{frame}{Outline}
    \tableofcontents
\end{frame}

\AtBeginSection[ ]
{
    \begin{frame}{Outline}
        \tableofcontents[currentsection]
    \end{frame}
}

\section{Introduction}

\begin{frame}
    \frametitle{Introduction}
    \begin{columns}
        \begin{column}{0.5\linewidth}
            \begin{itemize}
                \item Establishing and monitoring the intensity and duration of the
                    peak season of phenomenal is useful for certain Earth Observation
                    applications.
                \item For example, a precise estimation of the fire season in the
                    Amazon forest could improve resource allocation for fire brigades, 
                    better town planing in response to smoke, accurate estimations of
                    greenhouse gasses emissions.
                \item It is common to assume a fixed fire season for the Amazon
                    forest~\cite{carvalho2021}.
            \end{itemize}
        \end{column}
        \begin{column}{0.5\linewidth}
            \begin{figure}[h]
                \includegraphics[width=0.99\linewidth]
                {./images/corte_e_queima.png}
                \caption{Deforestation by slash and cut (\textit{Corte e queima}).
                Source:~\cite{dealmeida2022}.}
            \end{figure}
        \end{column}
    \end{columns}
\end{frame}

\begin{frame}
    \frametitle{Amazon fire calendar}
    \begin{columns}
        \begin{column}{0.5\linewidth}
            \begin{itemize}
                \item Stratification of the Amazon basin according to the dry
                    season start and \/end.
                \item It uses the mean monthly rainfall (CHIRPS) from 1989 to
                    2019 over a 10 km grid.
                \item The dry season is made of the consecutive months with 
                    rainfall below 100 mm.
                \item Regions are neighborhoods of pixels with the same start
                    and end.
                \item Their results are available online
                    \href{https://zenodo.org/records/5706455}{online}.
            \end{itemize}
        \end{column}
        \begin{column}{0.5\linewidth}
            \begin{figure}[h]
                \includegraphics[width=0.79\linewidth]
                {./images/carvalho2021.png}
                \includegraphics[width=0.69\linewidth]
                {./images/carvalho_fire_season_length.png}
                \caption{Source:~\cite{carvalho2021}.}
            \end{figure}
        \end{column}
    \end{columns}
\end{frame}

\section{Materials \& methods}

\begin{frame}
    \frametitle{Software}
    \begin{columns}
        \begin{column}{0.5\linewidth}
            \begin{itemize}
                \item R language~\cite{ihaka1996}.
                \item R packages \textit{dplyr} and \textit{ggplot2}.
                \item R packages for vector (\textit{sf}~\cite{pebesma2018})
                    and raster (\textit{terra}~\cite{hijmans2020}) data.
                \item R package \textit{sicegar} for double sigmoid
                    regression~\cite{caglar2018}.
                \item Analysis code available on
                    \href{https://github.com/albhasan/seasonmetrics}{GitHub}.
            \end{itemize}
        \end{column}
        \begin{column}{0.5\linewidth}
            \begin{figure}[h]
                \includegraphics[width=0.35\linewidth]{./logos/Rlogo.png}
                \includegraphics[width=0.29\linewidth]{./logos/sf_logo.png}
                \includegraphics[width=0.25\linewidth]{./logos/terra_logo.png}\\
                \includegraphics[width=0.24\linewidth]{./logos/dplyr.png}
                \includegraphics[width=0.24\linewidth]{./logos/ggplot2_logo.png}
            \end{figure}
        \end{column}
    \end{columns}
\end{frame}

\begin{frame}
    \frametitle{Data}
    \begin{itemize}
        \item We used fire data from VIIRS available at the INPE's "Queimadas"
            project.
    \end{itemize}
\end{frame}

\subsection{Method 1: Peak and threshold}

\begin{frame}
    \frametitle{Peak and threshold}.
    \begin{columns}
        \begin{column}{0.5\linewidth}
            \begin{itemize}
                \item Originally proposed by Guilherme Mataveli.
                \item A season is a subset of contiguous months that host the
                    peak and at least 60\% of the total intensity
                    (observations) of a phenomenon.
            \end{itemize}
        \end{column}
        \begin{column}{0.5\linewidth}
            \begin{figure}[h]
                \includegraphics[width=0.99\linewidth]
                {./images/peak_thres_hist.png}
            \end{figure}
        \end{column}
    \end{columns}
\end{frame}

\subsection{Method 2: Double sigmoidal}

\begin{frame}
    \frametitle{Double sigmoidal fitting}
    \begin{columns}
        \begin{column}{0.5\linewidth}
            \begin{itemize}
                \item Input data represents intensity measured over time.
                \item Growth happens in two phases: exponential intensity
                    increase until level off at a maximum level (first
                    sigmoid function); decay to a lower intensity or even zero
                    (second sigmoid).
                \item The midpoints are assumed as the start and end of the
                    season.
            \end{itemize}
        \end{column}
        \begin{column}{0.5\linewidth}
            \begin{figure}[h]
                \includegraphics[width=0.99\linewidth]
                {./images/dsig_function.png}
                \caption{Source:~\cite{caglar2018}.}
            \end{figure}
        \end{column}
    \end{columns}
\end{frame}

\section{Results}

\begin{frame}
    \frametitle{Results}
    \begin{figure}[h]
        \includegraphics[width=0.75\linewidth]
        {./images/pthres_vs_dsig.png}
    \end{figure}
\end{frame}

\begin{frame}
    \frametitle{Result comparison}
    \begin{itemize}
        \item Carvalho et al., ~\cite{carvalho2021} is actually about the
            establishing the dry and rainy seasons rather than the fire season.
        \item They use the fire spots to validate their results.
        \item Instead, we're using the fire spots to estimate the fire season
            and use ~\cite{carvalho2021}, to validate them.
    \end{itemize}
\end{frame}

\begin{frame}
    \frametitle{Result comparison}
    \begin{figure}[h]
        \includegraphics[width=0.75\linewidth]
        {./images/pthres_vs_carvalho2021.png}
    \end{figure}
\end{frame}

\section{Final remarks}

\begin{frame}
    \frametitle{Final remarks}
    \begin{itemize}
        \item Source code available at
                \url{https://github.com/albhasan/seasonmetrics}.
    \end{itemize}
\end{frame}

\begin{frame}[allowframebreaks]
    \frametitle{References}
    \bibliographystyle{amsalpha}
    \bibliography{seasonmetrics.bib}
\end{frame}

\end{document}
