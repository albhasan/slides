% NOTE:
%" latexmk -shell-escape -pvc slides.tex # Watches and compiles on each change.
% latexmk -c slides.tex   # Clean the temporal files.

% NOTE:
% the minted package doesn't play well with the bibliography!

\documentclass[aspectratio=169]{beamer}

\setbeamertemplate{footline}[frame number]

\usepackage{caption}
\usepackage{graphicx}
\usepackage{hyperref}
\usepackage{csvsimple}
\usepackage{booktabs}

\usefonttheme[onlymath]{serif}

\hypersetup{
    colorlinks=true,
    linkcolor=blue,
    filecolor=magenta,
    urlcolor=cyan,
    pdftitle={Overleaf Example},
    pdfpagemode=FullScreen,
}


\captionsetup[figure]{labelformat=empty}

\title{Methods for estimating the peak season in time series data}


\author{Alber S\'{a}nchez \href{mailto:alber.ipia@inpe.br}
{alber.ipia@inpe.br}\newline
Guilherme Mataveli}
\institute{
  \includegraphics[width=4cm,keepaspectratio]{logos/trees-color-h_2.png}
  \includegraphics[width=1.8cm,keepaspectratio]
  {logos/logoinpe-azul-menor.png} \\
  Research assistant - TreesLab\\National Institute for Space Research - INPE\\
  Brazil
}
\date{\today}

\begin{document}

\frame{\titlepage}

\begin{frame}{Outline}
    \tableofcontents
\end{frame}

\AtBeginSection[ ]
{
    \begin{frame}{Outline}
        \tableofcontents[currentsection]
    \end{frame}
}

\section{Introduction}

\begin{frame}
    \frametitle{Introduction}
    \begin{columns}
        \begin{column}{0.5\linewidth}
            \begin{itemize}
                    %What's known.
                \item Better estimations of the fire season in the Amazon
                    forest could foster better town planing and improve
                    responses to excessive fire smoke.
                    %What's unknown. Limitations and gaps in previous studies.
                \item Previous studies focused on the dry rather than the fire 
                    season and its regional patterns.
                \item Besides, it is common practice to assume a fixed fire
                    season.
                    %Burning question/hypothesis/aim.
                \item We present pixel-wise estimation of the fire season in
                    the Amazon based on fire spot detected by VIIRS.
                    %Why your experimental approach is new and different and important (fills in the gaps).
                    %Experimental approach.
                \item We developed a new method for estimating peak-seasons
                    given intensity data over time.
            \end{itemize}
        \end{column}
        \begin{column}{0.5\linewidth}
            \begin{figure}[h]
                \includegraphics[width=0.99\linewidth]
                {./images/corte_e_queima.png}
                \caption{Deforestation by slash and cut (\textit{Corte e queima}).
                Source:~\cite{dealmeida2022}.}
            \end{figure}
        \end{column}
    \end{columns}
\end{frame}

\begin{frame}
    \frametitle{Amazonian fire calendar}
    \begin{columns}
        \begin{column}{0.5\linewidth}
            \begin{itemize}
                \item Start, end, \& length of the dry season in the Amazon.
                \item Rain data (CHIRPS 1989-2019) on a 10Km grid.
                \item The dry season are the consecutive months with rainfall
                    below 100 mm.
                \item Regions are neighborhoods with the same season start \&
                    end.
                \item Critical fire periods are regions of similar dry season
                    determined by k-means clustering.
                \item Their results area available online for 
                    \href{https://zenodo.org/records/5706455}{download} and
                    \href{https://amazonianfirecalendar.shinyapps.io/fire_amazon}{visualization}.
            \end{itemize}
        \end{column}
        \begin{column}{0.5\linewidth}
            \begin{figure}[h]
                \includegraphics[width=0.79\linewidth]
                {./images/carvalho2021.png}
                \includegraphics[width=0.69\linewidth]
                {./images/carvalho_fire_season_length.png}
                \caption{Source:~\cite{carvalho2021}.}
            \end{figure}
        \end{column}
    \end{columns}
\end{frame}


\begin{frame}
    \frametitle{Justification}
    \begin{itemize}
        \item Fire in the Amazon is anthropogenic, not endemic.
        \item Fire is a cause and a tool in the deforestation and forest 
            degradation processes.
        \item An accurate fire calendar could improve decision-making and
            public policy design.
        \item A fire calendar could provide some required data for fire
            prevention and control.
    \end{itemize}
\end{frame}

\begin{frame}
    \frametitle{Research objective}
    \begin{itemize}
        \item Establish the beginning and end of the fire season in the Amazon
            rainforest with as much detail as available data allows.
    \end{itemize}
\end{frame}

\section{Materials \& methods}

\begin{frame}
    \frametitle{Software}
    \begin{columns}
        \begin{column}{0.5\linewidth}
            \begin{itemize}
                \item R language~\cite{ihaka1996}.
                \item R packages \textit{dplyr} and \textit{ggplot2}.
                \item R packages for vector (\textit{sf}~\cite{pebesma2018})
                    and raster (\textit{terra}~\cite{hijmans2020}) data.
                \item R package \textit{sicegar} for double-sigmoidal
                    regression~\cite{caglar2018}.
                \item Analysis code available on
                    \href{https://github.com/albhasan/seasonmetrics}{GitHub}.
            \end{itemize}
        \end{column}
        \begin{column}{0.5\linewidth}
            \begin{figure}[h]
                \includegraphics[width=0.35\linewidth]{./logos/Rlogo.png}
                \includegraphics[width=0.29\linewidth]{./logos/sf_logo.png}
                \includegraphics[width=0.25\linewidth]{./logos/terra_logo.png}\\
                \includegraphics[width=0.24\linewidth]{./logos/dplyr.png}
                \includegraphics[width=0.24\linewidth]{./logos/ggplot2_logo.png}
            \end{figure}
        \end{column}
    \end{columns}
\end{frame}

\begin{frame}
    \frametitle{Data}
    \begin{columns}
        \begin{column}{0.5\linewidth}
            \begin{itemize}
                \item We used 5 years of world fire data from VIIRS NPP (from
                    2019 to 2023, 10.213.4267 registries).
                \item We aggregated these data by month into a grid of 0.25$^{\circ}$.
            \end{itemize}
        \end{column}
        \begin{column}{0.5\linewidth}
            \begin{figure}
                \includegraphics[width=0.9\linewidth]{./logos/viirs.png}
            \end{figure}
        \end{column}
    \end{columns}
\end{frame}

\subsection{Method 1: Peak and threshold}

\begin{frame}
    \frametitle{Peak and threshold}.
    \begin{columns}
        \begin{column}{0.5\linewidth}
            \begin{itemize}
                \item Proposed by Guilherme Mataveli.
                \item A season is a subset of contiguous months that host the
                    peak and at least 60\% of the total intensity
                    (observations) of a phenomenon.
            \end{itemize}
        \end{column}
        \begin{column}{0.5\linewidth}
            \begin{figure}[h]
                \includegraphics[width=0.99\linewidth]
                {./images/peak_thres_hist.png}
            \end{figure}
        \end{column}
    \end{columns}
\end{frame}

\begin{frame}
    \frametitle{Peak and threshold example}
    \begin{columns}
        \begin{column}{0.2\linewidth}
            \begin{tabular}{cr}
                \toprule
                \bfseries Month & \bfseries GC
                \csvreader[head to column names]
                {./tables/monthly_counts.csv}{}
                { \\\Month & \GC }
                \\\bottomrule
            \end{tabular}
        \end{column}
        \begin{column}{0.8\linewidth}
            \begin{figure}[h]
                \centering
                \includegraphics[width=0.89\linewidth]
                {./images/peak_thres_example.png}
            \end{figure}
            \begin{table}
                \centering
            \begin{tabular}{ccccr}
                \toprule
                \bfseries Iteration & \bfseries Test Months & \bfseries Chosen & \bfseries Season & \bfseries Cum. Sum
                \csvreader[head to column names]
                {./tables/iteration.csv}{}
                { \\\Iteration & \TestMonths & \Chosen & \Season & \CumSum}
                \\\bottomrule
            \end{tabular}
            \end{table}
        \end{column}
    \end{columns}
\end{frame}

\subsection{Method 2: Double-sigmoidal}

\begin{frame}
    \frametitle{Double-sigmoidal fitting}
    \begin{columns}
        \begin{column}{0.5\linewidth}
            \begin{itemize}
                \item Input data represents intensity measured over time.
                \item Growth happens in two phases: exponential intensity
                    increase until level off at a maximum level (first
                    sigmoidal function); decay to a lower intensity or even zero
                    (second sigmoidal).
                \item The midpoints are assumed as the start and end of the
                    season.
            \end{itemize}
        \end{column}
        \begin{column}{0.5\linewidth}
            \begin{figure}[h]
                \includegraphics[width=0.99\linewidth]
                {./images/dsig_function.png}
                \caption{Source:~\cite{caglar2018}.}
            \end{figure}
        \end{column}
    \end{columns}
\end{frame}

\begin{frame}
    \frametitle{Logistic function}
    \begin{columns}
        \begin{column}{0.6\linewidth}
            \begin{equation}
                I(t) = f_{sig}(t) = \frac{I_{max}}
                { 1 + exp(-a_{1}(t - t_{mid}))}
            \end{equation}
        \end{column}
        \begin{column}{0.4\linewidth}
            Where~\cite{caglar2018}:
            \begin{itemize}
                \item $I(t)$ is the intensity as a function of time $t$.
                \item The parameters to fit are $I_{max}$, $t_{mid}$, and $a_{1}$.
                \item $I_{max}$ is the maximum intensity observed.
                \item $t_{mid}$ is the time at which intensity has reached half of its
                    maximum.
                \item $a_{1}$ is related to the slope of $I_{t}$ at $t = t_{mid}$ via 
                    $\frac{d}{dt} I(t)t = t_{mid} = a_{1}I_{max}/4$.
            \end{itemize}
        \end{column}
    \end{columns}
\end{frame}

\begin{frame}
    \frametitle{Double-sigmoidal function}
    Multipy two sigmoidal functions~\cite{caglar2018}:
    \begin{equation}
        f_{dsig-base}(t) = \frac{1}{1 + exp(-a'_{1}(t -t'_{mid1}))}
        \frac{1}{1 + exp(-a'_{2}(t -t'_{mid2}))}
    \end{equation}
    \begin{itemize}
        \item Let $t^*$ be the time at which $f_{dsig-base}(t)$ is maximal.
        \item Let $f_{max} = f_{dsig-base}(t^*)$.
        \item \begin{equation*}
                I(t) = f_{dsig}(t) = 
                \left\{
                    \begin{array}{rl}
                        c_{1}  f_{dsig-base}(t) \text{ for } t <= t^* \text{(growth phase)} \\
                        c_{2}  f_{dsig-base}(t) + I_{final} \text{ for } t > t^* \text{(decay phase)}
                    \end{array} \right.
        \end{equation*}
    \item $c_{1} = \frac{I_{max}}{f_{max}}$
    \item $c_{2} = \frac{(I_{max} - I_{final})}{f_{max}}$
    \end{itemize}
\end{frame}


\section{Results}

\begin{frame}
    \frametitle{Results}
    \begin{figure}[h]
        \includegraphics[width=0.80\linewidth]
        {./images/pthres_vs_dsig.png}
    \end{figure}
\end{frame}

\begin{frame}
    \frametitle{Fire \& dry seasons}
    \begin{itemize}
        \item Carvalho et al., ~\cite{carvalho2021} is actually about the
            establishing the dry season rather than the fire season.
        \item They use the fire spots to validate their results.
        \item Instead, we're using the fire spots to estimate the fire season
            and use ~\cite{carvalho2021}, to validate them.
    \end{itemize}
\end{frame}

\begin{frame}
    \frametitle{Fire \& dry seasons}
    \begin{figure}[h]
        \includegraphics[width=0.80\linewidth]
        {./images/pthres_vs_carvalho2021.png}
    \end{figure}
\end{frame}

\section{Final remarks}

\begin{frame}
    \frametitle{Final remarks}
    \begin{columns}
        \begin{column}{0.5\linewidth}
            \begin{itemize}
                \item We ran both methods for the world.
                \item Both peak \& threshold and double-sigmoidal methods can
                    be employed to estimate season of Earth Observation
                    phenomena besides fire.
                \item Source code available at
                    \url{https://github.com/albhasan/seasonmetrics}.
            \end{itemize}
        \end{column}
        \begin{column}{0.5\linewidth}
            \begin{figure}[h]
                \includegraphics[width=0.75\linewidth]
                {./images/pthres_world.png}
            \end{figure}
        \end{column}
    \end{columns}
\end{frame}

\begin{frame}
    \begin{figure}[h]
        \includegraphics[width=0.44\linewidth]
        {./images/pthres_world.png}
        \includegraphics[width=0.44\linewidth]
        {./images/dsig_world.png}
    \end{figure}
\end{frame}

\begin{frame}
    \frametitle{Next steps}
    \begin{itemize}
        \item Research question \textit{Is the fire season changing? If so, how?}
        \item Estimate the fire season over 20 years of data (using MODIS).
        \item Compute yearly deviations from the 20-year estimation. Choose
            particularly dry years.
        \item Check how well the fire season fits inside the dry season. Is
            the peak of the fire season inside the dry season? Is it centered?
        \item Generate figures and data to answer the research question.
    \end{itemize}
\end{frame}

\begin{frame}[allowframebreaks]
    \frametitle{References}
    \bibliographystyle{amsalpha}
    \bibliography{seasonmetrics.bib}
\end{frame}

\end{document}
