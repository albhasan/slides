\section{Material e methods}

For our analysis, we employed DETER data from August 2016 to July 2021. 
DETER is a real-time deforestation detection system developed by the National Institute of Space Research that is at the forefront of Brazil's efforts to control deforestation.
DETER has produced fast assessments of forest degradation and deforestation in the Brazilian Amazon since 2004~\cite{shimabukuro2006}. 
%TODO: Add statistics about DETER's dataset.

DETER mainly uses remote sensing imagery from the WFI camera on board of the CBERS satellites, producing alerts  of at least 3 ha which are tagged as wildfire scar, mining, deforestation with either exposed soil or vegetation, degradation, and selective cut with either disordered or geometric pattern~\cite{diniz2015a,f.g.assis2019}. 

Degradation and deforestation spots are identified by human experts on images enhanced with a color composition (red, near-infrared, and green) and a Linear Mixture Model~\cite{shimabukuro1991} (soil fraction) and the criteria of tone, color, shape, texture, and context. 
These experts draw DETER warnings (polygons) on top of a computer screen fix on a scale 1:100,000 using as background the latest primary forest polygon mask and previous DETER warnings~\cite{dealmeida2022}.
DETER data is publicly available at the TerraBrasilis portal~\cite{f.g.assis2019}. 

After downloading, we self-intersected the data (union operation) and re-projected them to the coordinate reference system UTM 22s.
We also removed duplicated vertices and enforced the right-hand rule for polygons.
Then we fixed geometry errors, and finally we removed alerts smaller than 3~ha.
This processing was applied using QGIS version 3.38.0~\cite{QGIS_software}.
We also computed the alerts' warning year using the PRODES calendar, which is the period from August to July; each PRODES year takes the year number from the last month of its period (July).

DETER alerts don't spatially match over time.
This means that it is only possible to match alert subareas consistently along the time dimension.
This is the reason for the data self-intersection mentioned above.
The \textit{subareas} resulting from this self-intersection correspond to polygons on which DETER alerts were issued on different dates, and are the base on which our analysis is founded.

Our analyses were carried out using the GNU's R language and environment for statistical computing and graphics to estimate statistics analysis~\cite{ihaka1996}.
Our source code is available online.~\footnote{R code available at \url{https://github.com/albhasan/treesburnareas}}
