\section{Discussion}

The data shows that recurrent DETER alerts over the same area are uncommon, and proportionally inverse to the number of alerts.
The data also shows that subareas with the most recurrent alerts tend to happen along the east of Amazonia (the deforestation arc) and to the north, in the \textit{Roraima} state (Figure~\ref{fig:plot_area_by_warnings} and Figure~\ref{fig:nwarnings_idw_map}).
Also, most of the successive alerts in the same subarea are at most five years apart (that is, the duration of our dataset, from 2016 to 2021), two years from the first to the second, and one year from there. 

It must be taken into account that DETER data is produced with for a different purpose than the analysis presented here, and that it acknowledges its under estimation of forest degradation due to its associated challenges~\cite{dealmeida2022}.
