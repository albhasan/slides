\section{Introduction}

% Background, known information.
The Amazon rainforest plays many roles in controlling the climate and avoiding the current climate crisis.
Not only it is home to countless species, but also regulates both the water and carbon cycles.
Besides, it is a massive carbon reservoir, and it is one of the potential tipping points at which poor management could trigger catastrophic and irreversible
changes to the climate system.

% Knowledge gap, unknown information.
The current advances in Ecology, Remote Sensing, and Computer Science have enabled the development of regional, continental and global deforestation monitoring systems (e.g., DETER, MapBiomas, Global Forest Watch).
However, detecting and monitoring forest degradation remains more challenging than detecting deforestation~\cite{Lambin:1999,Mitchell2017}.


% Hypothesis, question, purpose statement.
Given the importance of this issue, in this paper we present the spatial distribution of recurrent forest degradation in the Brazilian Amazon, which could help address the challenges in detecting it. 
We think that deforestation real-time forest monitoring systems, such as DETER that continuously issues deforestation alerts, inadvertently captures forest degradation processes at various stages.
% Approach, plan of attack, proposed solution.
The findings presented here are the results of processing 5 years of DETER 
alerts.

This paper extends the findings introduced in ~\cite{sanchez2023}.
