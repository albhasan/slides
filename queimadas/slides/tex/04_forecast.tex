\section{Forecast}

\begin{frame}
  Forecast
\end{frame}

\begin{frame}
  \frametitle{Forecast}
  \begin{figure}
    \centering
    \includegraphics[scale=0.5]{figures/plot_forecast_NPP-375D.png}
  \end{figure}
\end{frame}

\begin{frame}
  \frametitle{Forecast}
  \begin{figure}
    \centering
    \includegraphics[scale=0.5]{figures/plot_forecast_NPP-375-PM.png}
  \end{figure}
\end{frame}

\begin{frame}
  \frametitle{Forecast}
  \begin{figure}
    \centering
    \includegraphics[scale=0.5]{figures/plot_forecast_AQUA_M-T.png}
  \end{figure}
\end{frame}

\begin{frame}
  \frametitle{Forecast}
  \begin{figure}
    \centering
    \includegraphics[scale=0.5]{figures/plot_forecast_NOAA-12.png}
  \end{figure}
\end{frame}

\begin{frame}
    \frametitle{NOTES}
    \begin{itemize}
        \item The trends causes negative forecastings in NPP-357D and
            especially in NOAA-12. 
        \item The most well-behaved model correspond to the time series of
          NPP-375-PM.
        \item \textbf{We are considering NPP-375-PM's model the reference
          forecast}.
    \end{itemize}
\end{frame}
