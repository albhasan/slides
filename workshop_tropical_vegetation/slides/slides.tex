% NOTE:
% latexmk -shell-escape -pvc slides.tex # Compile PDF after any change.
% latexmk -c slides.tex                 # Clean temporal files.

\documentclass[aspectratio=169]{beamer}

\setbeamertemplate{footline}[frame number]
\beamertemplatenavigationsymbolsempty


% NOTE:Use numeric style for references.
\usepackage[backend=biber, style=numeric]{biblatex}
\usepackage{booktabs}
\usepackage{caption}
\usepackage{graphicx}
\usepackage{hyperref}
\usepackage{longtable}
\usepackage{minted}
\usepackage{siunitx}
\usepackage{subcaption}

\usetheme{Boadilla}
\usecolortheme{default}

\addbibresource{slides.bib}

\captionsetup[figure]{labelformat=empty}

\hypersetup{
colorlinks,
allcolors=.,
urlcolor=blue,
}

\title[Tower data]{Exploratory analisis of AmeriFlux data}
\subtitle{Advancements in tropical vegetation modelling and model-experiment 
integration}

%\author[Dutra, Débora]{D.~Dutra\inst{1} \and L.~Sato\inst{1} \and
\author[Dutra, Sato, Mataveli, Sãnchez]{D.~Dutra\inst{1} \and L.~Sato\inst{1} \and
G.~Mataveli\inst{1} \and A.~Sánchez\inst{1}}

\institute[INPE]{\inst{1} National Institute for Space Research}

\date{Freising, Germany\newline 2025-10-09}
\logo{
        \includegraphics[scale=0.19]{logos/logoinpe-azul-menor.png}
        \includegraphics[scale=0.07]{logos/logo_tum.png}
}



\begin{document}

\frame{\titlepage}

\begin{frame}
	\frametitle{Introduction}
	\begin{itemize}
		\item AmeriFlux and European flux databases.
		\item Only one file analyzed:
		      \textit{BR-Ma2\_HH\_199901010000\_201701010000\_LBAv2.csv}

	\end{itemize}
\end{frame}

\begin{frame}
	\frametitle{Method}
	\begin{itemize}
		\item Read data.
		\item Replace invalid data with NAs.
		\item Constrain analysis to only complete observations
		      occurring during August's daytime (08:00 - 18:00). August
		      is the fire peak season~\cite{carvalho2021a} and daylight,
		      which doesn't change much through the seasons near the
		      equator, is needed for photosynthesis.
		\item Produce a correlation matrix
		      (see Figure~\ref{fig:correlation_matrix}).
		\item Choose the most correlated variables.
		\item Plot observations and fit a quadratic model to the data
		      (see Figure~\ref{fig:point_cloud_plot}).
	\end{itemize}
\end{frame}

\begin{frame}
	\frametitle{Correlation matrix}
	\begin{figure}
		\begin{columns}
			\begin{column}{0.3\textwidth}
				\caption{This correlation matrix shows
					interessting variables for further
					analysis, such as \textit{FC},
					\textit{LW\_OUT}, \textit{RH}, and
					\textit{TA}. The actual correlation
					values were omitted to ease reading.}
				\label{fig:correlation_matrix}
			\end{column}
			\begin{column}{0.7\textwidth}
				\centering
				\includegraphics[scale=0.3]
				{figures/corr_plot.png}
			\end{column}
		\end{columns}
	\end{figure}
\end{frame}

\begin{frame}
	\frametitle{Most relevant variables}
	\begin{itemize}
		\item \textit{VPD\_F}: Vapor Pressure Deficit (\textit{hPA}).
		\item \textit{FC}: Carbon Dioxide (CO2) turbulence flux, no
		      storage correction ($\mu\ mol\ CO_{2}\ m^{-2}\ s^{-1}$).
		\item \textit{LW\_OUT}: Longwave radiation, outgoing
		      ($W\ m^{-2}$).
		\item \textit{RH}: Relative Humidity, range 0-100 (\%).
		\item \textit{TA}: Air temperature (\textit{deg C}).
	\end{itemize}
\end{frame}

\begin{frame}
	\frametitle{Observations' plot}
	\begin{figure}
		\begin{columns}
			\begin{column}{0.3\textwidth}
				\caption{This figure shows \textit{VPD\_F} as
					a function of the interesting variables
					selected before. Note the strong
					correlation of \textit{RH} and
					\textit{TA}.}
				\label{fig:point_cloud_plot}
			\end{column}
			\begin{column}{0.7\textwidth}
				\centering
				\includegraphics[scale=0.25]
				{figures/cloud_plot.png}
			\end{column}
		\end{columns}
	\end{figure}
\end{frame}

\begin{frame}
	\frametitle{Future work}
	\begin{itemize}
		\item Check the correlation details (e.g. $R^{2}$).
		\item Include relations to vegetation through indices, such as
		      NDVI, EVI, and GPP.
	\end{itemize}
\end{frame}

\begin{frame}[allowframebreaks]
	\frametitle{References}
	\printbibliography{}
\end{frame}

\end{document}
