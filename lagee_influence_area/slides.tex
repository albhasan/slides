% NOTE:
% latexmk -shell-escape -pvc slides.tex # Watches and compiles on each change.
% latexmk -c slides.tex   # Clean the temporal files.

% NOTE: 
% the minted package doesn't play well with the bibliography!

\documentclass[aspectratio=169]{beamer}

\setbeamertemplate{footline}[frame number]
\beamertemplatenavigationsymbolsempty

% NOTE: Only use numeric for references' style!
\usepackage[backend=biber,style=numeric]{biblatex}
\usepackage{booktabs}
\usepackage{caption}
\usepackage{graphicx}
\usepackage{hyperref}
\usepackage{siunitx}
\usepackage{subcaption}

\addbibresource{slides.bib}  

\captionsetup[figure]{labelformat=empty}

\title{How does LAGEE estimates areas of influence?}

\author{Alber S\'{a}nchez \href{mailto:alber.ipia@inpe.br}\newline
{alber.ipia@inpe.br}}
\institute{
  \includegraphics[width=1.8cm,keepaspectratio]
  {logos/logoinpe-azul-menor.png} \\
  Research assistant - TreesLab\\National Institute for Space Research - INPE\\
  Brazil
}
\date{\today}



\begin{document}



\frame{\titlepage}

\begin{frame}[allowframebreaks]
    \frametitle{Overview}
    \tableofcontents
\end{frame}



\section{Introduction}



 \begin{frame}
     Introduction.
 \end{frame}

 \begin{frame}
     \frametitle{About LaGEE}
     \begin{columns}
         \begin{column}{0.5\textwidth}
            \begin{itemize}
                \item Greenhouse Gas Laboratory (LaGEE) 
                    \url{https://www.ccst.inpe.br/lagee}
                \item Established in 2003.
                \item Part of the Earth System Science Center (CCST) at the 
                    National Institute for Space Research in Brazil (INPE).
                \item Principal Investigator: Dr. Luciana Gatti.
            \end{itemize}
         \end{column}
         \begin{column}{0.5\textwidth}
             \begin{figure}
                 \centering
                 \includegraphics[scale=.15]
                 {logos/logo-lagee-maiorzin.png}\newline
                 \includegraphics[scale=.17]
                 {logos/ccst.jpg}
                 \includegraphics[scale=.8]
                 {logos/logoinpe-azul-menor.png}
             \end{figure}
         \end{column}
     \end{columns}
 \end{frame}

\begin{frame}
    \frametitle{What are the LaGEE's Areas of Influcence?}
    Areas (or regions) of Influece (AI) are...
    \begin{itemize}
        \item Areas covered by the set of back-trajectories staring at 
            of the vertical profiles' sampling points over a time period.
        \item Derived from backtrajectory models forced by metereological input
            (mainly horizontal wind fields).
        \item AI are calculated from single back trajectories released from 
            each measurement point.
        \item Trajectories are binned to create an ensemble which is used to
            define an AI over most of the depth of the atmospheric profile over
            three months~\cite{cassol2020}.
    \end{itemize}
\end{frame}

\begin{frame}
    \frametitle{Uses of Areas of Influence}
    \begin{itemize}
        \item Combine and compare GHG flux estimates from upwind areas 
            to sample sampling sites of vertical atmospheric profiles.
        \item Determine the average value of metereological variables
            such as temperature and precipitation in the upwind area
            of a sampling sites of vertical profiles.
    \end{itemize}
\end{frame}

\begin{frame}
    \frametitle{Areas of influcence are NOT...}
    \begin{itemize}
        \item AI aren't intended to be used to calculate GHG fluxes.
        \item Areas of Influece are not Footprints:
        \begin{itemize}
            \item Footprints are determined from an ensemble of randomly 
                perturbated back-trajectories released at each measurement 
                point.
            \item Footprints are tipically used as sensitibily matrices in 
                Bayesian estimates of GHG fluxes in inversions.
            \item Footprints project fluxes into mole fraction 
                deviations ~\cite{cassol2020}.
            \item Footprints are defined in~\cite{gerbig2003,hu2019}.
        \end{itemize}
    \end{itemize}
\end{frame}

\begin{frame}
    \frametitle{What is a trajectory?}
    \begin{columns}
        \begin{column}{0.5\textwidth}
            \begin{itemize}
                \item A organized set of positions in space and time of a 
                    particule or an air parcel produce by an atmospheric model.
                \item A back-trajectory is a trajectory running on reverse, 
                    from the destination to the source.
            \end{itemize}
        \end{column}
        \begin{column}{0.5\textwidth}
            \begin{figure}
                \centering
                \includegraphics[scale=0.25]{img/example_trajectories.png}
                \caption{Trajectories ending at sampling site TAB on April 20th 
                2011. Source:~\cite{cassol2020}. }
            \end{figure}
        \end{column}
     \end{columns}
\end{frame}

\begin{frame}
    \frametitle{Hysplit}
    \begin{columns}
        \begin{column}{0.5\textwidth}
            \begin{itemize}
                \item Atmospheric transport and dispersion model.
                \item Compute simple air parcel trajectories.
                \item Complex transport, dispersion, chemical transformation,
                    and deposition simulations.
                \item Back trajectory analisys to determine the origin of air 
                    masses.
                \item Computation is an Lagrangian-Eulerian hybrid.
            \end{itemize}
        \end{column}
        \begin{column}{0.5\textwidth}
            \begin{figure}
                \centering
                \includegraphics[scale=0.15]
                {logos/noaa_air_resources_laboratory.png}
            \end{figure}
        \end{column}
    \end{columns}
\end{frame}

\begin{frame}
    \frametitle{Area of Influence - Method}
    \begin{itemize}
        \item Get trajectories from HySplit.
        \item Compute trajectory density using a one-degree grid covering 
            Brazil.
        \item Bin trajectories into 3-month intervals.
    \end{itemize}
\end{frame}

\begin{frame}
    \frametitle{Publication}
    \href{https://www.mdpi.com/2073-4433/11/10/1073}
    {DOI:10.3390/atmos11101073}~\cite{cassol2020}
    \begin{figure}
        \centering
        \includegraphics[scale=0.4]{img/cassol2020.png}
    \end{figure}
\end{frame}

\begin{frame}
    \frametitle{Areas of Influence - Results}
    \begin{figure}
        \centering
        \includegraphics[scale=0.07]{img/areas_of_influence.png}
        \caption{Areas of incluence. Source:~\cite{cassol2020}.}
    \end{figure}
\end{frame}



\section{How are IAs currently estimated?}



\begin{frame}
    How are the Areas of Influece currently estimated at LaGEE?
\end{frame}

\begin{frame}
    \frametitle{About LaGEE's code}
    \begin{columns}
        \begin{column}{0.5\textwidth}
            \begin{itemize}
                \item LaGEE has an online repository of code (click
                    ~\href{https://github.com/INPE-LAGEE}{here}).
                \item The code to process LaGEE's AIs is written as an 
                    \textit{R} package called \textit{cqmaTools}.
                \item \textit{cqmaTools} contains a directory for scripts, 
                    including the script for estimating the Areas of Influence.
                \item \textit{cqmaTools} depends on outdated packages 
                    (e.g. \textit{sp}), difficulting its installation.
            \end{itemize}
        \end{column}
        \begin{column}{0.5\textwidth}
            \begin{figure}
                \centering
                \includegraphics[scale=0.25]{img/github_lagee.png}
            \end{figure}
        \end{column}
    \end{columns}
\end{frame}

\begin{frame}[fragile]
    \frametitle{Script Areas of Influence}
    \begin{itemize}
        \item The actual script used for the AIs in LaGEE is different 
            from the one at GitHub.
        \item \citeauthor{cassol2020} states that ther are some IDL code
            involved during AI estimation. That code is missing from LaGEE's 
            code repository.
        \item The script analyzed here is called 
            \verb|influence_area_simuladas.R| and it was provided by Luciano 
            Marani on 2024/06/27.
    \end{itemize}
\end{frame}

\begin{frame}[fragile]
    \frametitle{Script \textit{influence\_area\_simuladas.R}}
    \begin{itemize}
        \item Dependencies:
            \begin{itemize}
                \item The development version of \textit{cqmaTools}.
                \item The deprecated \textit{raster} package.
            \end{itemize}
        \item Inputs:
            \begin{itemize}
                \item A path to a directory with trajectoy files.
                \item A path to a directory with station data. 
            \end{itemize}
        \item Outputs:
            \begin{itemize}
                \item Raster plot by site, by year, by trimester.
                \item A text with the data used for the plot.
            \end{itemize}
    \end{itemize}
\end{frame}

\begin{frame}[fragile,allowframebreaks]
    \frametitle{Processing in \textit{influence\_area\_simuladas.R}}
    \begin{itemize}
        \item Allow trajectories under 3500 m.
        \item Allow up to 4 missing neighbors for each center cell
            in a 3x3 window.
        \item Allow longitudes from -80 to -30.
        \item Allow latitudes from -40 to 10.
        \item Plot total:
            \begin{itemize}
                \item \verb|mean_trajectory|is the number of vertices in a site 
                    divided by the number of vertices in a year for that site.
                \item The threshold is 0.025
            \end{itemize}
        \item Plot by year:
            \begin{itemize}
                \item \verb|mean_trajectory| is the number of vertices in a 
                    year.
                \item The threshold is the maximum value of 
                    \verb|mean_trajectory| times the threshold (0.025).
            \end{itemize}
        \item Plot by trimester:
            \begin{itemize}
                \item \verb|mean_trajectory| is the number of vertices in a 
                    trimester.
                \item The threshold is the maximum value of 
                    \verb|mean_trajectory| times the threshold (0.025).
            \end{itemize}
        \item Plot by trimester (total):
            \begin{itemize}
                \item \verb|mean_trajectory| is the number of vertices in a 
                    trimester divided by the number of vertices in a trimester 
                    for that site.
                \item The threshold is the maximum value of 
                    \verb|mean_trajectory| times the threshold (0.025).
            \end{itemize}
    \end{itemize}
\end{frame}



\section{New proposal for estimating Areas of Influence}



\begin{frame}
    New proposal for estimating Areas of Influence.
\end{frame}

\begin{frame}[fragile]
    \frametitle{New proposal for estimating Areas of Influence}
    \begin{itemize}
        \item No smoothing.
        \item No fill in AI holes.
        \item Just comptute trajectories' vertex frequency over the grid.
        \item New script \verb|compute_frequency_grid.R|.
        % \item DEPRECATED: Only consider trajectories that don't go below 100 or above 1300 
        %     meters.
        % \item DEPRECATED: Keep trajectories that start in rnage 100 - 1300. 
        % \item DEPRECATED: Keep trajectories that remain 40\% in range 100 - 1300.
        % \item DEPRECATED: Consider the trajectory's segments in range 80 - 1300.
        % \item DEPRECATED: Do not consider trajectory's vertices below 250 meters. 
        \item Only consider points that don't go above 1300.
        \item Use trajectories' vertex count for each cell. Don't use the mean.
        \item Compute area of influence of each site for amazonia.
    \end{itemize}
\end{frame}



\section{Final remarks}



\begin{frame}
    Final remarks.
\end{frame}

\begin{frame}
    \frametitle{Take home message}
    \begin{columns}
        \begin{column}{0.7\textwidth}
            \begin{itemize}
                \item The LaGEE's R code is availabe at 
                    \url{https://github.com/INPE-LAGEE}.
                \item Get these slides at 
\footnotesize{\url{https://github.com/albhasan/slides/lagee_influence_area}}.
            \end{itemize}
        \end{column}
        \begin{column}{0.3\textwidth}
            \begin{figure}
                 \includegraphics[scale=2]
                 {logos/github-mark.png}
            \end{figure}
        \end{column}
    \end{columns}
\end{frame}

\begin{frame}[allowframebreaks]
	\frametitle{References}
	\printbibliography
\end{frame}

\end{document}

