\section{Material e methods}

For our analysis, we selected the municipality of \textit{São Félix do Xingu} 
in the Brazilian state of \textit{Pará} as our area of interest in the period 
from August 2016 to July 2021. 
This municipality extent is 82 square kilometers and it is consistently 
reported as one of the most deforested in 
Brazil by the PRODES system~\cite{f.g.assis2019}.

We prepare a dataset of degradation and deforestation warnings produced by the 
Brazilian National Institute for Space Research (INPE) and its DETER system.
DETER is a Geographic Information System which produces a fast assessment of 
forest degradation and deforestation in the Brazilian Amazon since 
2004~\cite{shimabukuro2006}. 
DETER is the backbone of law enforcement efforts in the Brazilian Amazon and 
Cerrado Biomes.
Since 2015, DETER uses remote sensing imagery captured by the WFI camera
on board of the CBERS 4 \& 4A satellites, producing warnings 
with a minimum area of 3 ha organized into classes: Wildfire scar, mining, 
deforestation with either exposed soil or vegetation, degradation, and 
selective cut with either disordered or geometric 
pattern~\cite{diniz2015,f.g.assis2019}. 
To spot deforestation, DETER employs human experts which use image composition 
of red, near-infrared, and green along with a Linear Mixture 
Model~\cite{shimabukuro1991} (soil fraction) and the criteria of tone, color, 
shape, texture, and context. 
These experts draw DETER warnings on top a computer screen fix on a scale 
1:100,000 using as background the latest PRODES primary forest mask and 
previous DETER warnings~\cite{dealmeida2022}.


Our data set consists of the warnings issued in our area of interest during the
aforementioned period downloaded from the TerraBrasilis 
portal~\cite{f.g.assis2019}. 
After downloading the data, we convert them into a 
GeoPackage;~\footnote{GeoPackage Encoding Standard 
\url{https://www.ogc.org/standards/geopackage}} this helps us overcome the 
limitations~\footnote{Switch from Shapefile 
\url{http://switchfromshapefile.org/}} of the traditional Shapefile format 
available from INPE. 

Then, it was used the self-intersect operation (union operation) on the warning 
polygons and after that, we projected them to the coordinate reference system 
UTM 22s; later, we removed duplicated vertices and enforced the right-hand-rule 
of polygons; then we applied successive geometry fixing, and finally we 
filtered out polygons smaller than 3 ha. 
These operations were applied using the software QGIS version 
3.28.0~\cite{QGIS_software}.
We also converted the warning year to what we call PRODES year, which is time 
period from August to July. Each PRODES year takes the year number from the 
last month of its period (July).

It is important to note that DETER warnings overlap over time but not
completely. That means, some segments of a warning partially overlap with 
some other segments of other warnings at other dates.
We call the result of the self-intersect operation \textit{subareas} and they 
correspond to segments of DETER warnings which, if overlap other subareas, 
they do have the same spatial properties (position, area, shape, and centroid).

Finally, we used the GNU's R language and environment for statistical 
computing and graphics to estimate statistics and carry on further 
analysis~\cite{r_manual}. 
The source code of our analysis is available 
online.~\footnote{File sbsr\_2013.R available at
\url{https://github.com/albhasan/treesburnareas}}


